\documentclass[11pt]{article}
\usepackage[utf8]{inputenc}

% page setup
\usepackage[letterpaper]{geometry}
\addtolength{\topmargin}{-0.7in}
\addtolength{\textheight}{1.8in}

\title{\bfseries%
Don't interpolate your data!}
\author{Hogg, Casey, Daunt, others?}
\date{October 2022}

\sloppy\sloppypar\raggedbottom\frenchspacing
\begin{document}

\maketitle

\begin{abstract}\noindent
When there are many observations of an astronomical source---many images with different dithers, or many spectra taken at different barycentric velocities---it is often tempting to shift and stack the data, to (for example) make a high signal-to-noise average image or mean spectrum.
Bound-saturating measurements are made not by manipulating data, but instead by optimizing (or otherwise using) 
a likelihood function, where the data are treated as fixed, and model parameters are modified to fit the data.
Traditional shifting and stacking of data can be converted into a model-fitting procedure, such that no data are ever harmed.
The key component of this conversion is a spectral model that is a continuous function of wavelength (or position in the case of imaging) that can represent the psf- and pixel-convolved signal being measured by the device after any reasonable translation.
The benefits of a modeling approach are myriad:
The sacred and expensive data never have to be interpolated or otherwise modified.
Noise maps, data gaps, and bad-data masks, none of which can easily be interpolated, don't have to be manipulated either.
The simplest model-fitting procedure, like the original shift-and-add procedure, is linear in the data, so noise propagation is straightforward.
The only cost is a small increase in computational complexity.
We demonstrate all these things with toy data; we provide open-source sample code for re-use.
\end{abstract}

\section{Introduction}

Hello World!

\section{Concepts and assumptions}

\paragraph{Multi-epoch spectra with shifts:}

\paragraph{Noise model:}

\paragraph{Bad-pixel mask:}

\paragraph{Data gaps:}

\paragraph{Pixel-convolved point-spread function:}

\paragraph{Continuous spectral model:}

\paragraph{Linear basis or mixture models:}

\paragraph{Non-uniform fast Fourier transform:}

\paragraph{Band limit:}

\paragraph{Known or fitted image offsets:}

\paragraph{Time variability:}

\paragraph{Sky, tellurics, flat-field, calibration:}

\section{Method}

\section{Toy experiments and results}

\section{Discussion}

Give some summary of the above.

Discuss time-variable inputs, and how you would modify to deal with that.

What about non-Gaussian noise?

What about data from different instruments with different LSFs / PSFs? What about data with different pixel scales?

\end{document}
